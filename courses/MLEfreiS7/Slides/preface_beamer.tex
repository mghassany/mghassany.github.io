
% theme metro and its options
\usetheme[progressbar=none,block=fill]{metropolis}
% \metroset{block=fill}

% \useoutertheme[subsection=false,shadow]{miniframes}

% \metroset{block=transparent} Options can be changed at any time — even mid-presentation! — with the \metroset macro.
\makeatletter
\setbeamertemplate{title page}{
  \begin{minipage}[b][\paperheight]{\textwidth}
    \usebeamercolor[fg]{section title}
  \usebeamerfont{section title}
    \centering  % <-- Center here
    \ifx\inserttitlegraphic\@empty\else\usebeamertemplate*{title graphic}\fi
    \vfill%
    \ifx\inserttitle\@empty\else\usebeamertemplate*{title}\fi
    \ifx\insertsubtitle\@empty\else\usebeamertemplate*{subtitle}\fi
    \usebeamertemplate*{title separator}
    \ifx\beamer@shortauthor\@empty\else\usebeamertemplate*{author}\fi
    \ifx\insertdate\@empty\else\usebeamertemplate*{date}\fi
    \ifx\insertinstitute\@empty\else\usebeamertemplate*{institute}\fi
    \vfill
    \vspace*{1mm}
  \end{minipage}
}


% \setbeamertemplate{section page}{
% \centering
% \begin{columns}
% \begin{column}{0.3\textwidth}
% \end{column}
% \begin{column}{0.7\textwidth}  %%<--- here
% \vspace*{1cm}
%   \begin{minipage}[c][\paperheight]{\textwidth}
%     \centering  % <-- Center here
%     \ifx\inserttitlegraphic\@empty\else\usebeamertemplate*{title graphic}\fi
%     \vfill%
%     \usebeamerfont{title}\insertsectionhead\par%
%     % \ifx\insertsubtitle\@empty\else\usebeamertemplate*{subtitle}\fi
%     \usebeamertemplate*{title separator}
%     \ifx\beamer@shortauthor\@empty\else\usebeamertemplate*{author}\fi
%     \ifx\insertdate\@empty\else\usebeamertemplate*{date}\fi
%     \ifx\insertinstitute\@empty\else\usebeamertemplate*{institute}\fi
%     \vfill
%     \vspace*{1mm}
%   \end{minipage}
% \end{column}
% \end{columns}
% }

% \setbeamertemplate{section page}{
% \begin{columns}
% \begin{column}{0.3\textwidth}
% \end{column}
% \begin{column}{0.7\textwidth}  %%<--- here
%   % \begin{center}
%   % \includegraphics[scale=0.15]{img/logos/Logo_ESILV.png}
%   % \end{center}
%   % \vfill%
%   \begin{minipage}[c]{0.9\textwidth}
%   \centering
%     \usebeamercolor[fg]{section title}
%   \usebeamerfont{section title}
%   
%   %\includegraphics[scale=0.18]{img/logos/Logo_ESILV.png}
%   
%   \vspace*{1cm}
%   
%   \insertsectionhead\\[-1ex]
%   
%   \vspace*{1cm}
%   
%   \ifx\inserttitle\@empty\else\usebeamertemplate*{title}\fi
%   \usebeamertemplate*{progress bar in section page}
%   \par
%     \ifx\beamer@shortauthor\@empty\else\usebeamertemplate*{author}\fi
%     \ifx\insertdate\@empty\else\usebeamertemplate*{date}\fi
%     \ifx\insertinstitute\@empty\else\usebeamertemplate*{institute}\fi
%     \vfill
%   \end{minipage}
% \end{column}
% \end{columns}
%   \par
%   \vspace{\baselineskip}
% }

\setbeamertemplate{title}{
%  \raggedright%  % <-- Comment here
  \linespread{1.0}%
  \inserttitle%
  \par%
  \vspace*{0.5em}
}
\setbeamertemplate{subtitle}{
%  \raggedright%  % <-- Comment here
  \insertsubtitle%
  \par%
  \vspace*{0.5em}
}

\setbeamertemplate{footline}
{
    \leavevmode%
    \hbox{%
        \begin{beamercolorbox}[wd=.2\paperwidth,ht=2.25ex,dp=1ex,left,leftskip=1ex]{title in head/foot}%
            \usebeamerfont{title in head/foot}\insertauthor
        \end{beamercolorbox}%
        \begin{beamercolorbox}[wd=.7\paperwidth,ht=2.25ex,dp=1ex,right,rightskip=1ex]{title in head/foot}%
            \usebeamerfont{title in head/foot}\insertsection
        \end{beamercolorbox}%
        \begin{beamercolorbox}[wd=.1\paperwidth,ht=2.25ex,dp=1ex,right]{date in head/foot}%
            \usebeamerfont{date in head/foot}
            \insertframenumber{} / \inserttotalframenumber\hspace*{2ex} 
        \end{beamercolorbox}}%
        \vskip0pt%
}
\makeatother

% \titlegraphic{%
%   \includegraphics[width=.3\textwidth]{img/logos/Logo_ESILV.png}
% }

\setbeamertemplate{frametitle}[default][center]

\usepackage{textpos}
\addtobeamertemplate{frametitle}{}{%
\begin{textblock*}{100mm}(-.05\textwidth,-0.7cm)
\includegraphics[height=3ex, keepaspectratio]{img/logo_hd_2.png}
\end{textblock*}}


%% fonts

\usefonttheme{professionalfonts}
% \usefonttheme{serif}
% \usefonttheme[onlymath]{serif}
\usepackage{eulervm}
%%packages

\usepackage{appendixnumberbeamer}

\usepackage[labelfont={color=barcolor,small},font={small,it}]{caption}

\usepackage{booktabs}
\usepackage[scale=2]{ccicons}

\usepackage{pgfplots}
\usepgfplotslibrary{dateplot}
 \usepackage{tikz}
 \usetikzlibrary{patterns}
% \usetikzlibrary{external}
% \tikzexternalize 

 
\usepackage{xspace}
\newcommand{\themename}{\textbf{\textsc{metropolis}}\xspace}

\usepackage{hyperref}

 \usepackage{lmodern}
 \usepackage{amssymb,amsmath,amsthm}
 \DeclareMathOperator*{\argmax}{argmax}
\DeclareMathOperator*{\argmin}{argmin}

\usepackage{bbm} % pour fonction indicatrice \mathbbm{1}
\usepackage{ifxetex,ifluatex}
\usepackage{color}
% \usepackage[dvipsnames]{xcolor}

\usepackage{cancel} % for strikethrough text in equation

% \usepackage{pgfplots}
\usepackage{caption,subcaption}
\usepackage{fontawesome}
% \usepackage{subfigure}
 \usepackage{fixltx2e} % provides \textsubscript
\ifnum 0\ifxetex 1\fi\ifluatex 1\fi=0 % if pdftex
  \usepackage[T1]{fontenc}
  \usepackage[utf8]{inputenc}
  %\usepackage [frenchb]{babel}
\else % if luatex or xelatex
  \ifxetex
    \usepackage{mathspec}
  \else
    \usepackage{fontspec}
  \fi
  \defaultfontfeatures{Ligatures=TeX,Scale=MatchLowercase}
\fi


\usepackage{mdframed}

% style
% \definecolor{barcolor}{HTML}{00B4A1}
% \definecolor{barcolor}{HTML}{CD0050} % ESILV COLOR

% \definecolor{barcolor}{HTML}{D11F5C} % ESILV COLOR
% \definecolor{backgroundcolor}{HTML}{ECECED} % ESILV COLOR

\definecolor{barcolor}{HTML}{077BBF} % EFREI COLOR
\definecolor{backgroundcolor}{HTML}{FEFEFE} %{ECECED} % ESILV COLOR
\definecolor{esilvcolor}{HTML}{077BBF}%{D11F5C}


% \setbeamercolor{palette primary}{fg=white,bg=barcolor}
% \usecolortheme[named=barcolor]{structure}
% \useoutertheme{smoothbars} % Alternatively: miniframes, infolines, split, smoothbars

\setbeamercolor{palette primary}{bg=barcolor,fg=backgroundcolor}
\setbeamercolor{palette secondary}{bg=barcolor,fg=backgroundcolor}
\setbeamercolor{palette tertiary}{bg=barcolor,fg=backgroundcolor}
\setbeamercolor{palette quaternary}{bg=barcolor,fg=backgroundcolor}
\setbeamercolor{structure}{fg=barcolor} % itemize, enumerate, etc
\setbeamercolor{section in toc}{fg=barcolor} % TOC sections
\setbeamercolor{background canvas}{bg=backgroundcolor}

% \setbeamercolor{background canvas}{bg=white!5}

\definecolor{green}{RGB}{0,158,115}

\setbeamercolor{alerted text}{ fg= esilvcolor , bg= backgroundcolor }
\setbeamercolor{example text}{ fg= green , bg= backgroundcolor }
\setbeamercolor{progress bar}{ fg= barcolor , bg= barcolor!20 }
% \setbeamercolor{title separator}{ ... }
% \setbeamercolor{progress bar in head/foot}{ ... }
% \setbeamercolor{progress bar in section page}{ ... }

\setbeamercolor{lower separation line foot}{bg=barcolor}

\setbeamertemplate{itemize item}{\scriptsize\color{barcolor}$\blacktriangleright$}
% \setbeamertemplate{itemize subitem}{\footnotesize\color{barcolor!50}$\centerdot
% $}
\setbeamertemplate{itemize subitem}[circle]

\addtobeamertemplate{block begin}
  {}
  {\vspace{0cm} % Pads top of block
     % separates paragraphs in a block
   % \setlength{\parskip}{24pt plus 1pt minus 1pt}
   }




% \addtobeamertemplate{block end}
%   {\vspace{2ex plus 0.5ex minus 0.5ex}}% Pads bottom of block
%   {\vspace{10ex plus 1ex minus 1ex}} % Seperates blocks from each other


%% Theorems, exercices, remarks

\newtheorem{theom}{\color{esilvcolor}Théorème}
\newtheorem{defi}{\textcolor{esilvcolor}{Définition}}[section]
% \theoremstyle{magentacolor} 
\theoremstyle{definition}
\newtheorem{prop}[theom]{\color{esilvcolor}Proposition} 
% \newtheorem*{solution}{Solution}
% \theoremstyle{definition}
\newtheorem*{preuve}{\textcolor{OliveGreen}{Démonstration}}
\newtheorem*{sol}{\textcolor{PineGreen}{Solution}}
\newtheorem*{re}{\textcolor{red}{Remarque}}
\newtheorem*{ex}{\textcolor{barcolor}{Exemple}}
\newtheorem*{propr}{\color{barcolor}Propriétés}


% \newtheoremstyle{exstyle}
%   {3pt} % Space above
%   {3pt} % Space below
%   {} % Body font
%   {} % Indent amount
%   {\color{Turquoise}\bfseries} % Theorem head font
%   {} % Punctuation after theorem head
%   {.5em} % Space after theorem head
%   {} % Theorem head spec (can be left empty, meaning `normal')

% \theoremstyle{exstyle}
% \newtheorem{ex}{Exemple}[section] 



\newcommand*{\QEDA}{\hfill\ensuremath{\blacksquare}}%
\newcommand*{\QEDB}{\hfill\ensuremath{\square}}%



%% shortcuts

\newcommand{\Om}{\Omega}
\newcommand{\om}{\omega}
\newcommand{\somme}{\sum_{i=1}^n}
\newcommand{\sommek}{\sum_{k=1}^n}
\newcommand{\sommekk}{\sum_{k=0}^n}


\newcommand{\intR}{\int_{-\infty}^{+\infty}}
\newcommand{\intab}{\int_a^b}
\newcommand{\intx}{\int_{-\infty}^x}
% \newcommand{}{}


% for the CDF of normal

\def\cdf(#1)(#2)(#3){0.5*(1+(erf((#1-#2)/(#3*sqrt(2)))))}%
% to be used: \cdf(x)(mean)(variance)

\DeclareMathOperator{\CDF}{cdf}

\newenvironment{de}{\metroset{block=fill}\begin{defi}}{\end{defi}} 
\newenvironment{theo}{\metroset{block=fill}\begin{theom}}{\end{theom}}


% \definecolor{metro}{HTML}{F9F9F9} %metro bg color 

% \useoutertheme[hideallsubsections,height=0pt]{sidebar}
% \setbeamercolor{sidebar}{bg=metro}
% \usecolortheme{sidebartab}
% \setbeamerfont{title in sidebar}{size=\fontsize{6}{8}\selectfont}
% \setbeamerfont{section in sidebar}{size=\fontsize{4}{6}\selectfont}

% \setbeamertemplate{navigation symbols}{%
% \insertslidenavigationsymbol
% \insertframenavigationsymbol
% \insertsectionnavigationsymbol
% \insertdocnavigationsymbol
% \insertbackfindforwardnavigationsymbol
% }

% to create columns in rmarkdown
\def\begincols{\begin{columns}}
\def\begincol{\begin{column}}
\def\endcol{\end{column}}
\def\endcols{\end{columns}}



\usepackage{tikz}
\usetikzlibrary{arrows,shapes}
\usetikzlibrary{mindmap} % to use grow cyclic
\usepackage{adjustbox}
\tikzstyle{block}=[draw opacity=0.7,line width=1.4cm]
\usetikzlibrary{bayesnet}


% R chunks
\makeatletter
\definecolor{shadecolor}{RGB}{236,236,237}
\makeatother

\definecolor{color1}{HTML}{E7AD00}
\definecolor{color2}{HTML}{A5CC19}
\definecolor{color3}{HTML}{33B29A}
\definecolor{color4}{HTML}{3380FF}
\definecolor{classificationblue}{HTML}{377EB8}

